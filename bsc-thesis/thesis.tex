\documentclass[a4paper,twoside,openright,11pt]{report}

\usepackage{amsmath}
\usepackage{amssymb}
\usepackage{amsthm}
\usepackage[labelfont=bf,labelsep=period]{caption}
\usepackage{enumitem}
\usepackage{float}
\usepackage[margin=2.5cm]{geometry}
\usepackage{graphicx}
\usepackage{hyperref}
\usepackage[utf8]{inputenc}
\usepackage{numbertabbing}    % This is a non-standard style.
\usepackage{times}
\usepackage{url}
\usepackage{xcolor}
\usepackage{xspace}
\usepackage{bm}
\usepackage{courier}

% for organising the document into multiple files
\usepackage{subfiles}

\floatstyle{ruled}
\newfloat{algo}{htbp}{algo}
\floatname{algo}{Algorithm}

% Fill in your data here.
\newcommand{\thesistitle}{3D Metric Fields}
\newcommand{\thesisauthor}{Florin Achermann}
\newcommand{\thesisauthororigin}{Bern, Switzerland}
\newcommand{\thesisleiter}{Prof.\ David Bommes}
% \newcommand{\thesisasst}{Denis Kalmykov}
\newcommand{\thesisurl}{http://cgg.unibe.ch/}
\newcommand{\thesissubtitle}{Optimizing Frame Fields in a new Metric}
\newcommand{\thesisdate}{15. September 2023}

% bars for matrices
\newcommand*{\vertbar}{\rule[-1ex]{0.5pt}{2.5ex}}
\newcommand*{\horzbar}{\rule[.5ex]{2.5ex}{0.5pt}}

%custom upright operators
\DeclareMathOperator{\vecop}{vec}
\DeclareMathOperator{\Id}{Id}
\newcommand{\op}[1]{\textsl{#1}}

\newtheorem*{remark}{Remark}


% gilles castel figures
\usepackage{import}
\usepackage{pdfpages}
\usepackage{transparent}
\usepackage{xcolor}
\newcommand{\incfig}[2][1]{%
    \def\svgwidth{#1\columnwidth}
    \import{./figures/}{#2.pdf_tex}
}
\pdfsuppresswarningpagegroup=1


\begin{document}

\pagenumbering{roman}

\begin{titlepage}  
  \thispagestyle{empty}

  \begin{center}  
    \begin{figure}[t]  
      \center{\includegraphics[scale=0.5]{UNI_Bern.png}}
      \vspace{1in}     
    \end{figure}
    
    {\bfseries\Huge \thesistitle \\[2mm]
      \Large \thesissubtitle}\\
    \vspace{1.5cm}

    {\bfseries\LARGE Bachelor Thesis}\\
    \vspace{1.5cm}
    
    {\Large \thesisauthor\\[2mm]
      from\\[2mm]
      \thesisauthororigin}\\
    \vspace{1.5cm}

    {\Large Faculty of Science, University of Bern}\\
    \vspace{1.5cm}

    {\Large \thesisdate}\\
    \vspace{1.5cm}

    \vspace*{\fill}
    {\Large
      \thesisleiter\\
      % \thesisasst\\
      Computer Graphics Group\\
      Institute of Computer Science\\
      University of Bern, Switzerland\\}
  \end{center}
\end{titlepage}


\chapter*{\centering Abstract}
\begin{quote}\noindent
  Automatic generation of volumetric hexahedral meshes continues
  to be a hard challenge. Directly optimizing for a decomposition 
  of an object into cube-like elements is unfeasible with todays
  available optimization techniques. Thus, this optimization problem is
  often relaxed through frame fields. By adding a covering
  metric field on the object to mesh and optimizing the frame field
  in this new metric, more control of the frame field is achieved.
  In this work, it is shown how to optimize a frame field in a
  new, user-defined metric. A full explanation is given of how to pull back
  the user-defined metric such that already available, euclidean metric frame field
  optimization techniques can be applied. The metric field is
  piece-wise linearly discretized, for which an exact and robust algorithm
  is formulated that finds all intersected tetrahedrons between two points in space.
  We test the optimization of frame field on some custom metric fields
  on cubes, where the resulting frame fields exhibit the expected singularity structure.

\end{quote}

\cleardoublepage


\tableofcontents


\cleardoublepage

\pagenumbering{arabic}

% main content
\subfile{introduction/introduction}
\subfile{chapter2/chapter2}
\subfile{chapter3/chapter3}
\subfile{chapter4/chapter4}
\subfile{chapter5/chapter5}
\subfile{chapter6/chapter6} % Simone algorithm
\subfile{experiments/experiments}

\chapter{Conclusion}
\label{ch:conclusion}
In this work, we showed how smoothness measured with covariant Dirichlet
energy is directly related to local integrability. In fact, we showed
that when the underlying metric field is flat, that local integrability
is actually equivalent to zero covariant derivative. This justifies
the application of metric fields to the optimization.
The approach is tested on cubes, where the resulting frame fields
look as expected.

To store the metric field, we used a tetrahedral mesh and piece-wise
linearly discretized the metric field. This is an improvement compared
to the constant discretization done in \emph{Metric-Driven 3D Frame Field Generation}\cite{Fang23}.
To calculate the rotation coefficients on the piece-wise linear metric field,
an exact and robust algorithm was needed to locate all tets between two points.
The formulation relies only on exact, arbitrary precision arithmetic to make decisions
and handles degeneracies, unlike the algorithm described in \emph{Walking on a Triangulation}\cite{Devillers}.
This is an original contribution.
To calculate the rotation coefficient, points are sampled recursively
until there is no noticable improvement.
There are minor things that impact speed but not
correctness which could be improved. Namely, the point localisation
in the $\op{tetFinder}$ could be changed to use e.g. k-d trees for
finding the nearest neighbor.


\paragraph{Future work.}
As an extension and to make this work more complete, several things
could be investigated.
First of, the frame fields get more accurate the more the voxels are
subdivided. It is thus natural to run the frame field optimization to
deeper depths. However, the current implementation on a standard office machine gets to depth $5$ in about 16 minutes,
which is pretty long for such a simple mesh like a cube.
As explained already, the original frame field optimization works by
adaptively refining voxels for higher resolution only where is needed.
Thus, the caching of the rotation coefficients in an Octree instead of 
a regular grid would be appropriate.

Currently, the metric fields have to be designed by hand and defined at
all vertices of the covering tetrahedral mesh everywhere.
This is a tedious process and not easy to do.
An automatic metric field design process where the user only inputs some
intuitive metric constraints at some vertices (e.g. ``larger'' metric at boundary, smaller metric
in the interior) and the vertices where no metric is defined would get
automatically assigned a metric that makes the metric field as easy to
optimize as possible would be of practical use.

Lastly, it is still unknown how much the piece-wise linear
discretization improved the accuracy of the computation of the rotation
coefficients.
A thorough comparison of results on the same models between the
constant discretization \cite{Fang23}
and linear discretization would paint a clearer picture if the extra
effort is worth it. 

 



\begin{itemize}
  \item we showed how smoothness is equivalent to local integrability
  when the metric is flat.
  \item we showed how to discretize linearly
  \item new exact and robust tetFinder algorithm
  \item improvement: higher depth, need to run optimization deeper
  \item improvement: speed and storage: the algorithms are currently
  formulated for correctness and simplicity. They could be improved by
  faster pointLocation. Storage of recursiveSubdivision coefficients

  Future work
  \item metric field generation (needs to be done by hand currently)
  \item investigation of linear discretization, how much better than piecewise constant
  \item how to store rotation coefficients in an octree
\end{itemize}


The conclusion looks back at the entire work, gives a critical look,
summarizes, and discusses extensions and future work.


\appendix
\chapter{Extra material}
\label{app:extra}

Extra material may be placed in an appendix that appears after the conclusion.


% The bibliography appears after any appendix.

% Remove this again once you understand how BiBTex citations work.
\nocite{*}

% Using BibTeX is required.
\bibliography{thesis}

% BibTeX entries are found in bibliography databases, here the file
% `thesis.bib`.  There is ample documentation available on the web on
% BibTeX.  Make sure that each entry is self-contained and that all entries
% are consistent.  This means, for instance, that an entry of type
% `article` contains all required fields (author, title, journal, year,
% volume) and some optional fields (number, pages, month, doi, note, key).
% The database is consistent when every entry of type `article` contains
% the same mandatory and optional fields.
%
% It is particularly easy to copy&paste BibTeX entries from DBLP, an online
% bibliography of computer science (https://dblp.org):
% (1) Find the author page (like https://dblp.org/pid/l/LeslieLamport.html);
% (2) navigate to the work to be cited ([b2]);
% (3) move the mouse over the vertical arrow and select
% "export record > BibTeX";
% (4) copy the record into your .bib file.
%
% If you want a DOI or URL to be shown with the standard bibstyles, then 
% change every field like
%   url       = {https://doi.org/10.1007/978-3-642-15260-3},
% to this
%   note      = {\url{https://doi.org/10.1007/978-3-642-15260-3}}
% .

% Any bibstyle is fine, this one give particularly compact output.
% This is a non-standard style.
\bibliographystyle{ieeesort}


% Mandatory declaration of origin, this must be the last page.
\chapter*{Erklärung}

\emph{Erklärung gemäss Art.~30 RSL Phil.-nat. 18}

\vspace*{1cm}

\noindent
Ich erkläre hiermit, dass ich diese Arbeit selbstständig verfasst und keine
anderen als die angegebenen Quellen benutzt habe. Alle Stellen, die
wörtlich oder sinngemäss aus Quellen entnommen wurden, habe ich als solche
gekennzeichnet. Mir ist bekannt, dass andernfalls der Senat gemäss Artikel
36 Absatz 1 Buchstabe r des Gesetzes vom 5. September 1996 über die
Universität zum Entzug des auf Grund dieser Arbeit verliehenen Titels
berechtigt ist.

\vspace*{1cm}

\noindent
Für die Zwecke der Begutachtung und der Überprüfung der Einhaltung der
Selbständigkeitserklärung bzw.  der Reglemente betreffend Plagiate erteile
ich der Universität Bern das Recht, die dazu erforderlichen Personendaten
zu bearbeiten und Nutzungshandlungen vorzunehmen, insbesondere die
schriftliche Arbeit zu vervielfältigen und dauerhaft in einer Datenbank zu
speichern sowie diese zur Überprüfung von Arbeiten Dritter zu verwenden
oder hierzu zur Verfügung zu stellen.

\vspace*{5cm}

\par\noindent\makebox[6cm]{\hrulefill}   \hfill\makebox[8cm]{\hrulefill}
\par\noindent\makebox[6cm][l]{Ort/Datum} \hfill\makebox[8cm][l]{Unterschrift}

\end{document}
