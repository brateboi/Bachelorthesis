\documentclass[../thesis.tex]{subfiles}

\begin{document}
\chapter{Mathematical Background}\label{ch:math-background}
We will make heavy use of differential geometry in the following chapters.
To get us all on the same page, I introduce the basic concepts what we will use, but I refrain from
giving any proofs. I will give definitions only as far as we need it. These definitions
will by no means be exhaustive. The following is an incomplete summary of what we need presented in
\emph{Introduction to smooth manifolds}\cite{Lee00}.

\paragraph{Manifold} A manifold $\mathcal{M}$ is a space that locally looks like Euclidean space.
More exactly, a $n$-manifold is a topological space, where each point on the manifold has an open neighborhood
that is locally homeomorphic to an open subset of Euclidean space $\mathbb{R}^n$.
A manifold can be equipped with additional structure. For example, we can work on \emph{smooth manifolds}.
In simple terms, a manifold is \emph{smooth} if it is similar enough to $\mathbb{R}^n$ that we can do Calculus
like differentiation or integration on it. For this, each point on the manifold must be
locally \emph{diffeomorphic} to an open subset of $\mathbb{R}^n$ space.

\paragraph{Tangent space, Tangent bundle} There are many equivalent definitions
for the tangent space. One definition is for each point $p$ in the manifold $\mathcal{M}$,
the tangent space $T_p\mathcal{M}$ consists of $\gamma'(0)$ for all differentiable paths $\gamma: (-\varepsilon, \varepsilon) \to \mathcal{M}$
with $p = \gamma(0)$. The tangent space is a vector space which has the same dimension as its manifold,
which is 3 in our case. These tangent spaces can be ``glued'' together to form the
\emph{tangent bundle} $T\mathcal{M} = \sqcup _{p \in \mathcal{M}}T_p\mathcal{M}$, which itself
is a manifold of dimension $2n$. An element of $T\mathcal{M}$ can be written
as $(p,v)$ with $p \in \mathcal{M}$ and $v \in T_p\mathcal{M}$. This admits a natural projection $\pi : T\mathcal{M} \to \mathcal{M}$,
which sends each vector $v \in T_p\mathcal{M}$ to the point $p$ where it is tangent: $\pi(p,v)=p$.
A \emph{section} $\sigma: \mathcal{M} \to T\mathcal{M}$ is a continuos map, with $\pi \circ \sigma = \Id_{\mathcal{M}}$.
Sections of $T\mathcal{M}$ are tangent vector fields on $\mathcal{M}$.

\paragraph{Cotangent space, Cotangent bundle} The dual space $V^*$ of a vector space $V$
consists of all linear maps $\omega: V \to \mathbb{R}$. We call these functionals \emph{covectors} on $V$.
$V^*$ is itself a vector space, with the same dimension as $V$ and operations like addition and scalar multiplication
can be performed on its elements. Any element in a vector space can be expressed as 
a finite linear combination of its basis. This basis is called the \emph{dual basis}.
Thus, we call the dual space of the vector space $T_p\mathcal{M}$ its \emph{cotangent space},
denoted by $T^*_p\mathcal{M}$. As before, the disjoint union of $T^*_p\mathcal{M}$ forms the \emph{cotangent bundle}:
$T^*\mathcal{M}=\sqcup _{p\in \mathcal{M}}T^*_p\mathcal{M}$. Defined analogously from above,
sections $\sigma$ on $T^*\mathcal{M}$ define \emph{covector fields} or \emph{1-forms}.

\paragraph{Tensors}
Before we can introduce differential forms in the next paragraph, we need to go a little bit into \emph{tensors}.
In simple words, tensors are real-valued, multilinear functions.
A map $F: V_1 \times \dots V_k \to W$ is multilinear, if $F$ is linear in each component.
For example, the dot product in $\mathbb{R}^n$ is a tensor. It takes two vectors and is linear in each component - bilinear.
Another example is the \emph{Tensor Product of Covectors}:
Let $V$ be a vector space and take two covectors $\omega, \eta \in V^*$.
Define the new function $\omega \otimes \eta: V\times V \to \mathbb{R}$ by
$\omega \otimes \eta (v_1,v_2) = \omega(v_1)\eta(v_2)$. It is multilinear, because $\omega$ and $\eta$ are linear.
We look at a special class of tensors, the \emph{alternating tensors}.
A tensor is alternating, if it changes sign whenever two arguments are interchanged,
i.e. $\omega(v_1, v_2) = -\omega(v_2, v_1)$.
A covariant tensor field over a manifold defines a covariant tensor at each point on the manifold,
covariant because the tensor is over the cotangent space $T^*_p\mathcal{M}$.
An alternating tensor field is called a \emph{differential form}.

\paragraph{Differential Forms, Exterior Derivative}
Recall that a section from $T^*\mathcal{M}$ is called a differential 1-form, or just 1-form.
Define the \emph{wedge product} (or \emph{exterior product}) between two 1-forms:
$$(\omega \wedge \eta)_p = \omega_p \wedge \eta_p$$
Notice the similarity to the \emph{Tensor Product of Covectors}: We get a new map, (a 2-form):
$$\omega \wedge \eta: T\mathcal{M} \times T\mathcal{M} \to \mathbb{R}$$
The wedge product is antisymmetric, therefore $\omega \wedge \eta = -\eta \wedge \omega$ for 1-forms $\omega$ and $\eta$.
We denote by $\Omega^k(\mathcal{M})$ the space of differential $k$-forms on $\mathcal{M}$.
There is a natural differential operator $d$ on differential forms we call \emph{exterior derivative}.
It maps $k$-forms to $(k+1)$-forms, i.e. $d: \Omega^k(\mathcal{M}) \to \Omega^{k+1}(\mathcal{M})$
and has the the following properties:
\begin{itemize}
  \item $df$ is the ordinary differential of a smooth function $f$. Smooth functions are 0-forms.
  \item $d(d\alpha) = 0$
  \item $d(\alpha \wedge \beta) = (d\alpha \wedge \beta) + (-1)^k(\alpha \wedge d\beta)$ for a $k$-form $\alpha$. (Leibnitz Rule)
\end{itemize}
In section \ref{ch:connection} we will need what $d\omega$ is for some smooth 1-form $\omega$, the calculation
will be done there. For now, just note that 
any arbitrary smooth 1-form can be written as $\omega = Fdx+Gdy+Hdz$ for some appropriate smooth functions $F,G,H$.


\paragraph{Riemannian metric, $g$-orthonormality}
Inner products are examples of symmetric tensors. They allow us to define lengths and angles
between vectors. We can apply this idea to manifolds.
A Riemannian metric $g$ is a symmetric positive-definite tensor field at each point.
If $\mathcal{M}$ is a manifold, the pair $(\mathcal{M},g)$ is called a \emph{Riemannian manifold}.
Let $g$ be the Riemannian metric on $\mathcal{M}$ and $p\in \mathcal{M}$,
then $g_p$ is an inner product on $T_p\mathcal{M}$. We write $\langle \cdot, \cdot\rangle_g$ to denote this inner product.
Any Riemannian metric can be written as positive-definite symmetric matrix, which allows for this simple form: $\langle v,w\rangle_g = v^{\top}gw$.

Such a new metric allows for the definition of \emph{$g$-orthonormality}:
A basis $[e_1, e_2, e_3]$ of $T_p\mathcal{M}$ is $g$-orthonormal if $\langle e_i, e_j \rangle_g = \delta_{ij}$.

\paragraph{Connection, Covariant derivative, Parallel transport}
A connection defines how two different tangent spaces are connected to each other, such that tangent vector fields
can be differentiated. There is an infinite amount of connections on a manifold.
An \emph{affine connection} $\nabla$ is a bilinear map that takes two tangent vector fields $X,Y$ and maps it
to a new tangent vector field $\nabla_XY$ on $\mathcal{M}$ such that
\begin{itemize}
  \item $\nabla_{fX}Y = f \nabla_XY$, where $f\in C^{\infty}(\mathcal{M}, \mathbb{R})$
  \item $\nabla_X(fY) = f\nabla_XY + (Xf)Y$ for $f\in C^{\infty}(\mathcal{M}, \mathbb{R})$, it satisfies the Leibnitz rule in the second variable
\end{itemize}
We call $\nabla_XY$ the \emph{covariant derivative of $Y$ in the direction of $X$}.
A connection $\nabla$ defines the parallel transport
of a vector along a curve. Given a curve $\gamma: [0,1] \to \mathcal{M}$ and
a vector $v_0 \in T_{\gamma(0)}\mathcal{M}$, there exists a unique parallel vector field $V: [0,1] \to T\mathcal{M}$ along $\gamma$
such that $V(0) = v_0$ \cite{LeeCurvature}.
Recall: a vector field $V$ along $\gamma$ means $\pi(V(t))=\gamma(t)$.
The uniqueness and existence is a consequence of the vector field $V$ being the solution
of $\nabla_{\dot{\gamma}(t)}V(t) = 0$ defining a linear ordinary differential equation with initial condition
$V(0)=v_0$. This vector field $V(t)$ is called the \emph{parallel transport} of $v_0$ along $\gamma$.
It is ``parallel'' in the sense that the transported vector does not change within the tangent space.
See figure \ref{fig:vectorfield} for an illustration in 2D.
\begin{figure}[htb]
  \centering
  \def\svgwidth{20em}
  \input{figures/vektorfeld.pdf_tex}
  \caption{A parallel vector field $V(t)$ along a curve $\gamma$. Each $V(t)\in T_{\gamma(t)}\mathcal{M}$ and
  $\nabla_{\dot{\gamma}(t)}V(t) = 0$, so each vector that is drawn is parallel to each other.}
  \label{fig:vectorfield}
\end{figure}
 


% \newpage
% A frame $F$ is a set of 6 vectors $\{\pm F_0, \pm F_1, \pm F_2 \}$.
% We can represent such a frame F as a $3\times3$ matrix F, where the $i$th-column is $F_i$.
% A frame field then maps to every point in 3D-space such a frame, i.e. $F: \mathbb{R}^3 \to \mathbb{R}^{3\times3}$.
% Usually, we work on a 3-manifold $\mathcal{M}$ and a positively oriented frame field, i.e.
% $F\vert_{\mathcal{M}}: \mathcal{M} \to \mathbb{R}^{3\times3}$, where $\det(F)>0$.
% To allow for anisotropic, nonuniform meshes, we generalize orthonormality of frames to
% $g$-orthonormal frames. Orthonormality is measured in some metric $g$, and a frame F satifisfies the condition
% $\langle F_i, F_j \rangle _g = \delta_{ij}$.
% Any frame field with $\det(F)>0$ naturally defines a metric $g= (FF^{\top})^{-1}$, where $F$ is $g$-orthonormal
% $$F^{\top}gF = Id.$$
% We can factor the frame field F into a symmetric part $g^{1/2}$ and a rotational part $R$
% $$F = g^{-1/2}R$$
% The symmetric part $g^{-1/2}$ keeps $F$ $g$-orthonormal
% $$ \implies F^{\top}gF = (g^{-1/2}R)^{\top}gg^{-1/2}R)=R^{\top}g^{-1/2}gg^{-1/2}R =Id.$$
% and $R$ represents a rotational field $R: \mathcal{M} \to SO(3)$.
% The requirements for our frame field are:
% \begin{itemize}
%   \item Smoothness
%   \item Integrability
%   \item Metric consistency: $g = (FF^{\top})^{-1}$
% \end{itemize}


\paragraph{Frame field, vector field, integrability}
Recall from the introduction that we are trying to find a hexahedral mesh
for some geometric object. In 2D, integer-grid maps have proven to be reliable
in generating the analogue of hexahedral meshes, quad meshes \cite{integer-grid}.
The idea is to use the same idea but in three dimensions.
The objective is to find a parametrization
$\phi : \mathcal{M} \to \mathbb{R}^3$, which embeds the $3$-dimensional object
onto an $3$-dimensional voxel grid, where the inverse $\phi^{-1}$ maps the deformed
grid back onto the object to recover a shape-aligned hexahedral mesh, see figure \ref{fig:integer-grid}.
\begin{figure}[htb]
  \centering
  \includegraphics[width=20em]{figures/integer-grid-rough}
  \caption{Idea of quadrangulation with integer-grid maps in 2D}
  \label{fig:integer-grid}
\end{figure}
This is called an integer-grid map and by minimizing distortion and satisfying boundary alignment,
a hexahedral mesh could be extracted.
However, this is a hard mixed-integer and non-convex optimization problem for which no current
optimization technique is available that would result in an acceptable solution.
By taking the Jacobian of the parametrization, $\nabla \phi$, we get a mapping
$\nabla \phi : \mathcal{M} \to \mathbb{R}^{3\times3}$ of \emph{frames}, a \emph{frame field}.
The idea of frame fields is then to search for an approximation of $\nabla \phi$.
If $F: \mathcal{M} \to \mathbb{R}^{3\times3}$ is the approximation of $\nabla \phi$, we can
solve for $\phi : \mathcal{M} \to \mathbb{R}^3$ with
$$\min \limits_{\phi} \int \limits_{\mathcal{M}} ||\nabla \phi F - \Id||^2$$
where $||\cdot||$ is the Frobenius norm. If this is sufficiently small, then $F^{-1} \approx \nabla \phi$ and 
the extracted hexahedral mesh closely follow the integer-grid isolines.
We treat a frame $F$ as a set of 3 linearly independent vectors $\{F_1,F_2,F_3 \}$ which we can
collect into a matrix $F=(F_1,F_2,F_3) \in \mathbb{R}^{3\times 3}$.
A frame locally represents the edges of a deformed cube.
We can think of a frame field as the composition of three vector fields. However, frame fields
can contain singularities that the underlying vector fields do not contain \cite{Nieser}.
Notice that many of these frames are equivalent.
There are $2^3$ choices of the sign, and $3!$ possible permutations, which
gives $2^3\cdot 3! = 48$ equivalent frames.
Since we want non-degenerate frames and the same orientation through the grid, the constraint $\det(F)>0$ is imposed,
which leaves $24$ equivalent frames.
Equivalence of frames is then defined as
$$F_u \sim F_v \iff \exists R \in \mathcal{O} : F_u=RF_v$$
where $\mathcal{O}$ is the \emph{chiral cubical symmetry group}\cite{Nieser}.
These symmetries make the optimization for frame fields more complicated.
To handle these symmetries, we use the spherical harmonics based representation \cite{Huang}.
The idea is to use the polynomial $x^4+y^4+z^4$ to express the frames as rotations.
Under the restriction of the polynomial to the sphere, that is $\mathbb{S}^2\to \mathbb{R}$, the octahedral frames
are invariant under the chiral cubical symmetry group. However, the spherical harmonics
manifold is 9-dimensional so not all vectors in the spherical harmonics represent valid frames,
since rotations only exhibit 3 degrees of freedom.
While optimizing the frame field, some kind of ``average'' of two frames will be done.
This averaging makes sense in the spherical harmonics space, but the result may be outside the valid
space of frames. A projection from the spherical
harmonics space back to the valid space of frames will be needed, see fig. \ref{fig:projection} \cite{Ray}.
\begin{figure}[htb]
  \centering
  \def\svgwidth{20em}
  \input{figures/projection.pdf_tex}
  \caption{Combination of two valid frames leads to a resulting frame outside the valid frames. A projection to the nearest valid frame is done.}
  \label{fig:projection}
\end{figure}


When a frame $F$ has orthonormal columns, the resulting hex elements look like unit cubes.
By introducing a metric $g$, we can control the size and shear of the cubes by
relaxing the orthonormality constraint to $g$-orthonormality when
$$F^{-1}gF = \Id$$
holds. We can decompose the frame field $F$ into a rotational part $R : \mathcal{M} \to \text{SO}(3)$ and a symmetric metric part $g^{-1/2}$
(akin to the polar decomposition of linear transformations)\cite{Panozzo}, that is $F = g^{-1/2}R$.

\begin{figure}[htb]
  \centering
  \includegraphics[width=25em]{figures/factorization}
  \caption{Factorization of a $g$-orthogonal frame field into a symmetric metric part $g^{-1/2}$ and a rotational part $R$
  (Figure from \cite{Fang23}).}
  \label{fig:factorization}
\end{figure}

\begin{itemize}
  \item Dirichlet energy, what are we optimizing
  \item Requirements for the frame field -> integrability + boundary alignment
\end{itemize}




\end{document}