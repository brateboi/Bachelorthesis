\documentclass[../thesis.tex]{subfiles}

\begin{document}
\chapter{Mathematical Background}\label{ch:math-background}
We will make heavy use of differential geometry in the following sections.
To get us all on the same page, I introduce the basic concepts what we will use, but I refrain from
giving any proofs. I will give definitions only as far as we need it. These definitions
will by no means be exhaustive. The following is an incomplete summary of what we need presented in
``Introduction to smooth manifolds'' \cite{Lee00}

\paragraph{Manifold} A manifold $\mathcal{M}$ is a space that locally looks like Euclidean space.
More exactly, a $n$-manifold is a topological space, where each point on the manifold has an open neighborhood
that is locally homeomorphic to an open subset of Euclidean space $\mathbb{R}^n$.
A manifold can be equipped with additional structure. For example, we can work on \emph{smooth manifolds}.
In simple terms, a manifold is \emph{smooth} if it is similar enough to $\mathbb{R}^n$ that we can do Calculus
like differentiation or integration on it. For this, each point on the manifold must be
locally \emph{diffeomorphic} to an open subset of $\mathbb{R}^n$ space.

\paragraph{Tangent space, Tangent bundle} There are many equivalent definitions
for the tangent space. One definition is for each point $p$ in the manifold $\mathcal{M}$,
the tangent space $T_p\mathcal{M}$ consists of $\gamma'(0)$ for all differentiable paths $\gamma: (-\varepsilon, \varepsilon) \to \mathcal{M}$
with $p = \gamma(0)$. The tangent space is a vector space which has the same dimension as its manifold,
which is 3 in our case. These tangent spaces can be ``glued'' together to form the
\emph{tangent bundle} $T\mathcal{M} = \sqcup _{p \in \mathcal{M}}T_p\mathcal{M}$, which itself
is a manifold of dimension $2n$. An element of $T\mathcal{M}$ can be written
as $(p,v)$ with $p \in \mathcal{M}$ and $v \in T_p\mathcal{M}$. This admits a natural projection $\pi : T\mathcal{M} \to \mathcal{M}$,
which sends each vector $v \in T_p\mathcal{M}$ to the point $p$ where it is tangent: $\pi(p,v)=p$.
A \emph{section} $\sigma: \mathcal{M} \to T\mathcal{M}$ is a continuos map, with $\pi \circ \sigma = \Id_{\mathcal{M}}$.
Sections of $T\mathcal{M}$ are vector fields on $\mathcal{M}$.

\paragraph{Cotangent space, Cotangent bundle} The dual space $V^*$ of a vector space $V$
consists of all linear maps $\omega: V \to \mathbb{R}$. We call these functionals \emph{covectors} on $V$.
$V^*$ is itself a vector space, with the same dimension as $V$ and operations like addition and scalar multiplication
can be performed on its elements. Any element in a vector space can be expressed as 
a finite linear combination of its basis. This basis is called the \emph{dual basis}.
Thus, we call the dual space of the vector space $T_p\mathcal{M}$ its \emph{cotangent space},
denoted by $T^*_p\mathcal{M}$. As before, the disjoint union of $T^*_p\mathcal{M}$ forms the \emph{cotangent bundle}:
$T^*\mathcal{M}=\sqcup _{p\in \mathcal{M}}T^*_p\mathcal{M}$. Defined analogously from above,
sections $\sigma$ on $T^*\mathcal{M}$ define \emph{covector fields} or \emph{1-forms}.

\paragraph{Tensors}
Before we can introduce differential forms in the next paragraph, we need to go a little bit into \emph{tensors}.
In simple words, tensors are real-valued, multilinear functions.
A map $F: V_1 \times \dots V_k \to W$ is multilinear, if $F$ is linear in each component.
For example, the dot product in $\mathbb{R}^n$ is a tensor. It takes two vectors and is linear in each component - bilinear.
Another example is the \emph{Tensor Product of Covectors}:
Let $V$ be a vector space and take two covectors $\omega, \eta \in V^*$.
Define the new function $\omega \otimes \eta: V\times V \to \mathbb{R}$ by
$\omega \otimes \eta (v_1,v_2) = \omega(v_1)\eta(v_2)$. It is multilinear, because $\omega$ and $\eta$ are linear.
We look at a special class of tensors, the \emph{alternating tensors}.
A tensor is alternating, if it changes sign whenever two arguments are interchanged,
i.e. $\omega(v_1, v_2) = -\omega(v_2, v_1)$.
A covariant tensor field over a manifold defines a covariant tensor at each point on the manifold,
covariant because the tensor is over the cotangent space $T^*_p\mathcal{M}$.
An alternating tensor field is called a \emph{differential form}.

\paragraph{Differential Forms, Exterior Derivative}
Recall that a section from $T^*\mathcal{M}$ is called a differential 1-form, or just 1-form.
Define the \emph{wedge product} (or \emph{exterior product}) between two 1-forms:
$$(\omega \wedge \eta)_p = \omega_p \wedge \eta_p$$
Notice the similarity to the \emph{Tensor Product of Covectors}: We get a new map, (a 2-form):
$$\omega \wedge \eta: T\mathcal{M} \times T\mathcal{M} \to \mathbb{R}$$
The wedge product is antisymmetric, therefore $\omega \wedge \eta = -\eta \wedge \omega$ for 1-forms $\omega$ and $\eta$.
There is a natural differential operator $d$ on differential forms we call \emph{exterior derivative}.
The exterior derivative is a generalization of the differential of a function.
In particular, a smooth function $f$ (a 0-form) has the derivative $df$ which is a 1-form.
The exact definition is not important for us, so let us just look at some properties so we can work with it.
If $\omega$ is a k-form, $d\omega$ is a (k+1)-form.
TODO

\paragraph{Riemannian metric, $g$-orthonormality}
Inner products are examples of symmetric tensors. They allow us to define lengths and angles
between vectors. We can apply this idea to manifolds.
A Riemannian metric $g$ is a symmetric positive-definite tensor field at each point.
If $\mathcal{M}$ is a manifold, the pair $(\mathcal{M},g)$ is called a \emph{Riemannian manifold}.
Let $g$ be the Riemannian metric on $\mathcal{M}$ and $p\in \mathcal{M}$,
then $g_p$ is an inner product on $T_p\mathcal{M}$. We write $\langle \cdot, \cdot\rangle_g$ to denote this inner product.
Any Riemannian metric can be written as positive-definite symmetric matrix, which allows for this simple form: $\langle v,w\rangle_g = v^{\top}gw$.

Such a new metric allows for the definition of \emph{$g$-orthonormality}:
A basis $[e_1, e_2, e_3]$ of $T_p\mathcal{M}$ is $g$-orthonormal if $\langle e_i, e_j \rangle_g = \delta_{ij}$.

\paragraph{connection, covariant derivative}
\paragraph{lie algebra so(3)}
\paragraph{Frame field, vector field, integrability}


\newpage
A frame $F$ is a set of 6 vectors $\{\pm F_0, \pm F_1, \pm F_2 \}$.
We can represent such a frame F as a $3\times3$ matrix F, where the $i$th-column is $F_i$.
A frame field then maps to every point in 3D-space such a frame, i.e. $F: \mathbb{R}^3 \to \mathbb{R}^{3\times3}$.
Usually, we work on a 3-manifold $\mathcal{M}$ and a positively oriented frame field, i.e.
$F\vert_{\mathcal{M}}: \mathcal{M} \to \mathbb{R}^{3\times3}$, where $\det(F)>0$.
To allow for anisotropic, nonuniform meshes, we generalize orthonormality of frames to
$g$-orthonormal frames. Orthonormality is measured in some metric $g$, and a frame F satifisfies the condition
$\langle F_i, F_j \rangle _g = \delta_{ij}$.
Any frame field with $\det(F)>0$ naturally defines a metric $g= (FF^{\top})^{-1}$, where $F$ is $g$-orthonormal
$$F^{\top}gF = Id.$$
We can factor the frame field F into a symmetric part $g^{1/2}$ and a rotational part $R$
$$F = g^{-1/2}R$$
The symmetric part $g^{-1/2}$ keeps $F$ $g$-orthonormal
$$ \implies F^{\top}gF = (g^{-1/2}R)^{\top}gg^{-1/2}R)=R^{\top}g^{-1/2}gg^{-1/2}R =Id.$$
and $R$ represents a rotational field $R: \mathcal{M} \to SO(3)$.
The requirements for our frame field are:
\begin{itemize}
  \item Smoothness
  \item Integrability
  \item Metric consistency: $g = (FF^{\top})^{-1}$
\end{itemize}



\end{document}