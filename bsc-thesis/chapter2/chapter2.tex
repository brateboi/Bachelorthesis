\documentclass[../thesis.tex]{subfiles}

\begin{document}
\chapter{Mathematical Background}
We will make heavy use of differential geometry in the following sections.
To get us all on the same page, I introduce the basic concepts what we will use, but I refrain from
giving any proofs. I will give definitions only as far as we need it. These definitions
will by no means be exhaustive. The following is an incomplete summary of what we need presented in
``Introduction to smooth manifolds'' \cite{Lee00}

\paragraph{Manifold} A manifold $\mathcal{M}$ is a space that locally looks like Euclidean space.
More exactly, a $n$-manifold is a topological space, where each point on the manifold has an open neighborhood
that is locally homeomorphic to an open subset of Euclidean space $\mathbb{R}^n$.
A manifold can be equipped with additional structure. For example, we can work on \emph{smooth manifolds}.
In simple terms, a manifold is \emph{smooth} if it is similar enough to $\mathbb{R}^n$ that we can do Calculus
like differentiation or integration on it. For this, each point on the manifold must be
locally \emph{diffeomorphic} to an open subset of $\mathbb{R}^n$ space.

A \emph{Riemannian manifold} $(\mathcal{M},g)$ is a real, smooth manifold, which is additionally equipped with
a metric $g$ at each point $p$ on the manifold.

\paragraph{Tangent space, Tangent bundle} There are many equivalent definitions
for the tangent space. One definition is for each point $p$ in the manifold $\mathcal{M}$,
the tangent space $T_p\mathcal{M}$ consists of $\gamma'(0)$ for all differentiable paths $\gamma: (-\varepsilon, \varepsilon) \to \mathcal{M}$
with $p = \gamma(0)$. The tangent space is a vector space which has the same dimension as its manifold,
which is 3 in our case. These tangent spaces can be ``glued'' together to form the
\emph{tangent bundle} $T\mathcal{M} = \sqcup _{p \in \mathcal{M}}T_p\mathcal{M}$, which itself
is a manifold of dimension $2n$. An element of $T\mathcal{M}$ can be written
as $(p,v)$ with $p \in \mathcal{M}$ and $v \in T_p\mathcal{M}$. This admits a natural projection $\pi : T\mathcal{M} \to \mathcal{M}$,
which sends each vector $v \in T_p\mathcal{M}$ to the point $p$ where it is tangent: $\pi(p,v)=p$.
A \emph{section} $\sigma: \mathcal{M} \to T\mathcal{M}$ is a continuos map, with $\pi \circ \sigma = \Id_{\mathcal{M}}$.
Sections of $T\mathcal{M}$ are vector fields on $\mathcal{M}$.

\paragraph{Cotangent space, Cotangent bundle} The dual space $V^*$ of a vector space $V$
consists of all linear maps $L: V \to \mathbb{R}$.

\paragraph{Metric}
\paragraph{g-orthonormality}
\paragraph{Exterior Derivative, Differential}
\paragraph{One-Forms in particular}
\paragraph{connection, covariant derivative}
\paragraph{lie algebra so(3)}
\paragraph{Frame field, vector field, integrability}


\newpage
A frame $F$ is a set of 6 vectors $\{\pm F_0, \pm F_1, \pm F_2 \}$.
We can represent such a frame F as a $3\times3$ matrix F, where the $i$th-column is $F_i$.
A frame field then maps to every point in 3D-space such a frame, i.e. $F: \mathbb{R}^3 \to \mathbb{R}^{3\times3}$.
Usually, we work on a 3-manifold $\mathcal{M}$ and a positively oriented frame field, i.e.
$F\vert_{\mathcal{M}}: \mathcal{M} \to \mathbb{R}^{3\times3}$, where $\det(F)>0$.
To allow for anisotropic, nonuniform meshes, we generalize orthonormality of frames to
$g$-orthonormal frames. Orthonormality is measured in some metric $g$, and a frame F satifisfies the condition
$\langle F_i, F_j \rangle _g = \delta_{ij}$.
Any frame field with $\det(F)>0$ naturally defines a metric $g= (FF^{\top})^{-1}$, where $F$ is $g$-orthonormal
$$F^{\top}gF = Id.$$
We can factor the frame field F into a symmetric part $g^{1/2}$ and a rotational part $R$
$$F = g^{-1/2}R$$
The symmetric part $g^{-1/2}$ keeps $F$ $g$-orthonormal
$$ \implies F^{\top}gF = (g^{-1/2}R)^{\top}gg^{-1/2}R)=R^{\top}g^{-1/2}gg^{-1/2}R =Id.$$
and $R$ represents a rotational field $R: \mathcal{M} \to SO(3)$.
The requirements for our frame field are:
\begin{itemize}
  \item Smoothness
  \item Integrability
  \item Metric consistency: $g = (FF^{\top})^{-1}$
\end{itemize}



\end{document}