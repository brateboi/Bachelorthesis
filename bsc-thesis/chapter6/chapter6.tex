\documentclass[../thesis.tex]{subfiles}
\begin{document}

\chapter{Frame Field optimization}
\label{ch:optimization}
Until now, we have only covered how we measure Dirichlet energy $||\mathcal{D}R||$ in
the new metric $g$.
This chapter covers how we minimize
$$E(\mathcal{M})=\int_{\mathcal{M}}||DR||^2.$$
We use an unpublished scheme from \emph{[Simone Raimondi]} based
on the Merriman-Bence-Osher (MBO) algorithm.
In principle, any frame field optimization scheme that works based on
optimizing the Dirichlet energy (or its discretized version $||R(q)-R(p)||^2$) can be
modified with the rotation coefficient $R_{qp}$ to optimize in a new metric.
\section{Optimization Algorithm}
The algorithm works by dividing the manifold into cubes,
and optimizing the frames per cube. The algorithm works based on an adaptive
grid, refining the grid and optimizing where more resolution is needed, see figure \ref{fig:grid} 
\begin{figure}[htb]
    \centering
    \def\svgwidth{20em}
    \input{figures/grid.pdf_tex}
    \caption{An adaptive grid is formed around the manifold (bold red), with increased resolution where needed.
    The rotation coefficients for a given voxel are calculated from its neighbours (arrows).
    In 3D, the cubes get divided into eight smaller cubes.}
    \label{fig:grid}
\end{figure}
From a given voxel with coordinates $v=(x,y,z)$, the rotation coefficients
are calculated from their neighbors $v_n=(x\pm 1,y\pm 1,z\pm 1)$ to $v$,
i.e. $$||[R_{v_n\leadsto v}]v_n - v||^2.$$
The MBO algorithm is based on repeated diffusion in the spherical harmonics (as averaging frames
makes sense in the spherical harmonics), with a projection back to the manifold
of valid frames.
\begin{figure}[htb]
    \centering
    \def\svgwidth{20em}
    \input{figures/diffusionprojection.pdf_tex}
    \caption{Repeated diffusion of frames in spherical harmonics with a projection back to the
    valid space of frames is used to minimize the energy. Figure inspired from \cite{Palmer} }
    \label{fig:diffusionprojection}
\end{figure}


Because the implementation for the adaptivity of the grid was not finished
at the time of this thesis, the optimization is done on a regular grid where
all voxels have the same size. Two parameters depth \texttt{d} and number of diffusion steps \texttt{n}
are present. The depth $d$ is how many times the grid gets subdivided, with depth \texttt{0}
corresponding to a cube that is subdivided once.
The resulting frame fields are expected to be the same, it just does it slower.

\section{Caching of coefficients}

\begin{itemize}
    \item Problem statement
    \item Simone algorithm from a high level
    \item Caching of coefficients
\end{itemize}

\end{document}