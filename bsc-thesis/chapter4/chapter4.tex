\documentclass[../thesis.tex]{subfiles}



\begin{document}

% These macros are only used for the pseudo-code example.
\newcommand{\str}[1]{\textsc{#1}}
\newcommand{\var}[1]{\textit{#1}}
\newcommand{\op}[1]{\textsl{#1}}
\def \ifempty#1{\def\temp{#1} \ifx\temp\empty }
\newcommand{\msg}[2]{\ensuremath{\ifempty{#2} [\str{#1}] \else [\str{#1}, {#2}] \fi}}
\newcommand{\tup}[1]{\ensuremath{\langle #1 \rangle}}
\newcommand{\nil}{\ensuremath{\bot}}
\newcommand{\false}{\textsc{false}\xspace}
\newcommand{\true}{\textsc{true}\xspace}

\newcommand\lastts{\var{lastts}\xspace}
\newcommand\nextts{\var{nextts}\xspace}
\newcommand\trusted{\var{trusted}\xspace}
\newcommand\newepoch{\var{newepoch}\xspace}
\newcommand\leader{\var{leader}\xspace}
\newcommand\ts{\var{ts}\xspace}
\newcommand{\CK}{\ensuremath{\mathcal{K}}\xspace}
\newcommand{\CP}{\ensuremath{\mathcal{P}}\xspace}
\newcommand{\CQ}{\ensuremath{\mathcal{Q}}\xspace}


\chapter{Calculation of rotation coefficient}
\label{ch:calculation}
% Open Question from me: In the paper, they say that in general $R_{\ell}$ cannot be simplified
% to $\exp (-\int_{\ell}\omega)$, because of the noncommutativity of rotations in 3D.
% I say we can, because of the closedness of our one-form, therefore the
% integral is path-independent.
We are interested in the rotation.
Let $p,q$ be points on the manifold and $\ell$ a path connecting them.
Let $R: \mathcal{M} \to SO(3)$ be a rotation field, i.e. $R(p), R(q)$ are orthogonal frames.
Under the initial condition that $R(p)= \Id$, the equation
$$R(q)= \exp(R_{pq}(\omega))R(p)$$
is a differential equation that corresponds to the parallel transport under the connection $\omega$
of the frame along $\ell$.
To recover this rotation $R_{pq}$, we integrate $\omega$ along $\ell$.
\paragraph{Discretization}
We parametrize the path by $\ell(0)=a, \ell(1)=b$.
We resort to numerical integration for $R_{ab}$ and cut the path into $n$ small segments, i.e.
$$R_{ab}= R_nR_{n-1} \dotsb R_1$$
where $R_i = \exp(-\omega (\dot{\ell}(i\gamma))\gamma) = \exp((\int W^{\top}dp)_{\times})$, and $\gamma$ is the length of a segment and $\dot{\ell}(s)=\frac{\partial \ell}{\partial s}(s)$.
Calculating the exponential map of an antisymmetric matrix (which $\omega \in \mathfrak{so}(3)$ is) can be done with Rodrigues' formula:
$$\exp(u_{\times}) = \Id + \sin(\theta) \hat{u}_{\times} + (1-\cos(\theta)) \hat{u}_{\times}^2$$
where $\theta = ||u||_2$ is the rotation angle and $\hat{u}= u/\theta$ is the rotation axis.
We use the trapezoidal rule to evaluate the a short interval of the integral of $W$, which is given by
$$R_{ab}=\exp \left( \left(\frac{1}{2}(W_a + W_b)^{\top}(b-a)\right)_{\times}\right)$$
where $W_a, W_b$ is solved for by the 9x9 linear system given by $A_x$ and $\nabla \times A$.


\paragraph{Piecewise linear discretization}
We discretize our metric field with a tetrahedral mesh $\mathcal{T}$.
At each vertex, we attach a metric and linearly interpolate with barycentric coordinates within a tet.

Let $A_i \in \mathbb{R}^{3x3}, i \in \{1,2,3,4\}$ be the square root metrics at the vertices $v_i \in \mathbb{R}^3$ of a tet, such that
$A^2_i = g(v_i)$. We represent a tet given by its four vertices by a 3x4 matrix, i.e.
$$\begin{pmatrix}
  \vertbar & \vertbar & \vertbar & \vertbar \\
  v_1 & v_2 & v_3 & v_4 \\
  \vertbar & \vertbar & \vertbar & \vertbar
\end{pmatrix} \in \mathbb{R}^{3\times 4}.$$
Any point $p$ within the tet can then be represented as
$$p = \alpha v_1 + \beta v_2 + \gamma v_3 + \delta v_4$$
with $\alpha, \beta, \gamma, \delta \ge 0$ and $\alpha + \beta + \gamma + \delta = 1$.
This is a linear transformation between two coordinate systems, which we can write in matrix form as
$$\underbrace{\begin{pmatrix}
  \vertbar & \vertbar & \vertbar & \vertbar \\
  v_1 & v_2 & v_3 & v_4 \\
  \vertbar & \vertbar & \vertbar & \vertbar \\
  1 & 1 & 1 & 1
\end{pmatrix}}_{T^{-1}}
\underbrace{
\begin{pmatrix}
  \alpha \\ \beta \\ \gamma \\ \delta
\end{pmatrix}}_{\lambda} = \underbrace{\begin{pmatrix}
  x \\ y \\ z \\ 1 
\end{pmatrix}}_{p}
\iff
T^{-1}\lambda = p \iff  \lambda = Tp
$$
$T$ always exists because $v_1, \dots , v_4$ are linearly independent,
else it would not be a tetrahedron.
By denoting $T= \{ t_{ij} \}_{i,j \in \{ 1,\dots, 4\} }$, we can write our barycentric functions as
\begin{align*}
\alpha(x,y,z) &= t_{11}x + t_{12}y + t_{13}z + t_{14} \\
\beta(x,y,z) &= t_{21}x + t_{22}y + t_{23}z + t_{24} \\
\gamma(x,y,z) &= t_{31}x + t_{32}y + t_{33}z + t_{34} \\
\delta(x,y,z) &= t_{41}x + t_{42}y + t_{43}z + t_{44}
\end{align*}
The convex combination
$$A(x,y,z) = \alpha A_1 + \beta A_2 + \gamma A_3 + \delta A_4$$
is the metric prescribed in the tet. To find $\nabla \times A$, let
$A = (A^1, A^2, A^3)$. We will need the derivatives for the curl, so let
$$(A^i_j)_x \triangleq \frac{\partial A^i_j}{\partial x}$$ be the derivative with respect to $x$ of entry $i,j$.
E.g. $(A^i_j)_x$ is given by
$$(A^i_j)_x = \alpha_x (A_1)^i_j + \beta_x (A_2)^i_j + \gamma_x (A_3)^i_j + \delta_x (A_4)^i_j = t_{11}(A_1)^i_j + t_{21}(A_2)^i_j + t_{31}(A_3)^i_j + t_{41}(A_4)^i_j$$
If we write $T= (T^1, T^2, T^3, T^4)$ and collect $(A_k)^i_j$ into a vector
$$\bar{A^i_j}=\begin{pmatrix}
  (A_1)^i_j \\ (A_2)^i_j \\ (A_3)^i_j \\ (A_4)^i_j
\end{pmatrix}$$
this can be shortened to $\bar{A^i_j}^{\top}T^1 = (A^i_j)_x$.
Analogously, we get
$$(A^i_j)_y = \bar{A^i_j}^{\top}T^2  \text{ and } (A^i_j)_z = \bar{A^i_j}^{\top}T^3$$
The curl is then given by
$$\nabla \times A^i = \begin{pmatrix}
  (A^i_3)_y - (A^i_2)_z \\
  (A^i_1)_z - (A^i_3)_x \\
  (A^i_2)_x - (A^i_1)_y
\end{pmatrix} =
\begin{pmatrix}
  \bar{A^i_3}^{\top}T^2 - \bar{A^i_2}^{\top}T^3\\
  \bar{A^i_1}^{\top}T^3 - \bar{A^i_3}^{\top}T^1\\
  \bar{A^i_2}^{\top}T^1 - \bar{A^i_1}^{\top}T^2\\
\end{pmatrix}$$
and $\nabla \times A = \nabla \times (A^1, A^2, A^3)$.
Notice that the curl is constant within a tetrahedron.

\paragraph*{Recursive Subdivision}
Whenever the rotation coefficient $R_{pq}$ is needed between some points $p$ and $q$,
it is unclear ahead of time how many sampling points on the line $\ell$
are needed such that $R_{pq}$ accurately describes how the frame rotates along $\ell$.
We apply a recursive subdivision scheme to recursively sample more points on $\ell$ only where is needed
until sampling more points leads to no noticable improvement anymore (see figure \ref{fig:recursiveSubdivision}).
We begin by calculating $R_{ab}$ with just the endpoints. This coefficient is then compared to the result if the midpoint was sampled aswell,
so i.e. if
$$\frac{||R_{ab} - R_{am}\cdot R_{mb}||^2_2}{\ell^2}< \varepsilon$$
then no measurable improvement happened. We divide by the length of the segment $\vec{ab}$, because the segments get smaller
and we want the tolerance $\varepsilon$ to remain the same.
This approach has the advantage of only sampling points where there is improvement.
See section \ref{ch:experiments} for results.
\begin{figure}[htb]
  \centering
  \def\svgwidth{20em}
  \input{figures/recursiveSubdivision.pdf_tex}
  \caption{Points are recursively sampled as midpoints between the line $\vec{ab}$.
  If sampling more points within some line segment leads to noticable improvement, more points are sampled.}
  \label{fig:recursiveSubdivision}
\end{figure}

The following code does what is described above.
\begin{algo*}
  \vbox{
    \small
    \begin{numbertabbing}
      \reset
      xxxx\=xxxx\=xxxx\=xxxx\=xxxx\=xxxx\=MMMMMMMMMMMMMMMMMMM\=\kill
      \textbf{recursiveDivide} $(a,b)$ \label{}\\
      \> $R_1 = \op{Rotation(a,b)}$ \label{}\\
      \> $\ell = \op{length}(a-b)$ \label{}\\
      \> midpoint = $\frac{a+b}{2}$\label{}\\
      \> \textbf{if} $\frac{||R_1 - Rotation(a,midpoint)\cdot Rotation(midpoint,b)||^2_2}{\ell^2} < \varepsilon$ \label{}\\
      \> \> return $R_1$ \label{}\\
      \> \textbf{else} \label{}\\
      \> \> return $\op{recursiveDivide}(a,midpoint) \cdot \op{recursiveDivide}(midpoint, b)$ \label{}\\
    \end{numbertabbing}
  }
  \caption{Recursive Subdivision}
  \label{alg:recursiveSubdivision}
\end{algo*}


\end{document}

