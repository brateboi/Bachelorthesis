\documentclass[../thesis.tex]{subfiles}



\begin{document}

% These macros are only used for the pseudo-code example.
\newcommand{\str}[1]{\textsc{#1}}
\newcommand{\var}[1]{\textit{#1}}
\def \ifempty#1{\def\temp{#1} \ifx\temp\empty }
\newcommand{\msg}[2]{\ensuremath{\ifempty{#2} [\str{#1}] \else [\str{#1}, {#2}] \fi}}
\newcommand{\tup}[1]{\ensuremath{\langle #1 \rangle}}
\newcommand{\nil}{\ensuremath{\bot}}
\newcommand{\false}{\textsc{false}\xspace}
\newcommand{\true}{\textsc{true}\xspace}

\newcommand\lastts{\var{lastts}\xspace}
\newcommand\nextts{\var{nextts}\xspace}
\newcommand\trusted{\var{trusted}\xspace}
\newcommand\newepoch{\var{newepoch}\xspace}
\newcommand\leader{\var{leader}\xspace}
\newcommand\ts{\var{ts}\xspace}
\newcommand{\CK}{\ensuremath{\mathcal{K}}\xspace}
\newcommand{\CP}{\ensuremath{\mathcal{P}}\xspace}
\newcommand{\CQ}{\ensuremath{\mathcal{Q}}\xspace}


\chapter{Alignment of frames}
\label{ch:calculation}
When the metric $g$ is globally flat with no additional constraints like
boundary alignment, a globally integrable frame field is achievable
by parallel transport. No singularities in the frame field would be present.
However, this is not possible in general.
In this section, by starting from the assumption that the metric
is flat, we recover the rotation happening by parallel transport under
the connection $\omega$ which allows us
to pull back the covariant derivative based Dirichlet energy
and minimize as if we were minimizing
the Dirichlet energy in Euclidean coordinates.

The smoothness measure $\mathcal{D}R$ is zero if all frames are parallel
to each other under the connection $\omega$, i.e., if all frames
are the same if parallel transported to the same point.
This is deliberately kept vague as the parallel transport is path
dependent in general (parallel transport is path independent if the metric is flat).
To get a feeling for the problem statement, see Figure \ref{fig:rotation}.

\begin{figure}[htb]
  \centering
  \def\svgwidth{20em}
  \input{figures/rotation.pdf_tex}
  \caption{Under the metric induced by $\phi$, the frame at $q$ is parallel to $p$.
  To compare them, we need to recover how $p$ is rotated under $\omega$.}
  \label{fig:rotation}
\end{figure}

As we want parallelness under $\omega$ along a path $\gamma$, we set $\mathcal{D}R \overset{!}{=}0$ and treat it as a set of
differential equations. Again, as $R$ essentially consists of 
three rotation fields, what is meant that $\mathcal{D}R_i =0$ for each field.

Let $p,q$ be points on the manifold and $\ell : [0,1]\to \mathcal{M}$ a path connecting them,
with $\ell(0)=q$ and $\ell(1)=p$.
Our differential equation is then given by
\begin{equation}
  \bm{0} = dR_i(\ell(t)) + \omega(\ell(t)) R_i(\ell(t))
\end{equation}
which we rearrange to
\begin{equation}
  \underbrace{-\omega (\ell(t))}_{\bm{A}(t)}  \underbrace{R_i(\ell(t))}_{\bm{\alpha}(t)} = \underbrace{dR_i(\ell(t))}_{\bm{\dot{\alpha}}(t)}
\end{equation}
where we explicitly indicate that $\omega$ depends on the point evaluated and is not constant.
Do not get confused what we are differentiating:
The differential equation consists of mappings 
$$\omega : \mathcal{M} \to \mathbb{R}^{3\times3} \text{ and }R_i : \mathcal{M}\to \mathbb{R}^3.$$
Thus, the concatenation with $\ell: [0,1] \to \mathcal{M}$ gives us mappings
$\bm{A}: [0,1] \to \mathbb{R}^{3\times 3}$ and $\bm{\alpha}:[0,1] \to \mathbb{R}^3$
where it makes sense to take the derivative with respect only to $t$.
This differential equation has the form of a \emph{matrix differential equation},
for which a general closed form solution is hard to find.
However, when $\omega$ and its integral $\int_0^t-\omega(\ell(s))ds$ commute, the general solution
$\bm{\alpha}(t)= e^{\int_0^t \bm{A}(s)ds}\bm{\alpha}(0)$ exists,
which is the case when $\omega$ along $\ell$ rotate around the same axis
(the same way rotation matrices in 3D only commute when they rotate around the same axis).
Plugging in values $0$ and $1$ for $t$ and assuming $\omega$ commutes with its integral gives the solution
$$\bm{\alpha}(1) = R(\ell(1)) = R(p) = e^{\int_0^1\bm{A}(s)ds}R(\ell(0))= e^{\int_0^1\bm{A}(s)ds}R(q),$$
which shows that we can 
align frames with a single rotation matrix $R_{qp}$, i.e.
$$R(p)= R_{qp}R(q)$$
which corresponds to the parallel transport under the connection $\omega$
of the frame along $\ell$.
Discretization of the covariant based Dirichlet energy is then given by
\begin{equation}\label{eq:dirichlet}
  ||\mathcal{D}R||_{\mathrm{discretized}}=||R_{qp}R(q)-R(p)||^2.
\end{equation}
We parametrize the path by $\ell(0)=a, \ell(1)=b$.
We resort to numerical integration for $R_{ab}$ and cut the path into $n$ small segments, i.e.
$$R_{ab}= R_nR_{n-1} \dotsb R_1$$
where $R_i = \exp(-\omega (\dot{\ell}(i\gamma))\gamma)$, $\gamma$ is the length of a segment and $\dot{\ell}(s)=\frac{\partial \ell}{\partial s}(s)$.
By cutting up the path into small segments, $\omega$ should rotate around the same axis along $\ell$, which as described above
allows us to use the general solution $R_{i}=\exp((-\int_{\gamma} \omega d\bm{p})_{\times})$.
Calculating the exponential map of an antisymmetric matrix (which $\omega \in \mathfrak{so}(3)$ is) yields a
rotation matrix $R\in \mathrm{SO}(3)$ and can be done with Rodrigues' formula:
$$\exp(u_{\times}) = \Id + \sin(\theta) \hat{u}_{\times} + (1-\cos(\theta)) \hat{u}_{\times}^2$$
where $\theta = ||u||_2$ is the rotation angle and $\hat{u}= u/\theta$ is the rotation axis.
We use the trapezoidal rule to evaluate the short interval of the integral of $W$, which is given by
$$R_{ab}=\exp \left( \left(\frac{1}{2}(W_a + W_b)^{\top}(b-a)\right)_{\times}\right)$$
where $W_a, W_b$ are solved for by the $9\times9$ linear system given by $A_{\times}$ and $\nabla \times A$ from equation \ref{eq:linearsystem}.


\section{Piecewise linear discretization}
We discretize our metric field with a tetrahedral mesh $\mathcal{T}$.
At each vertex, we attach a metric and linearly interpolate with barycentric coordinates within a tet.

Let $A_i \in \mathbb{R}^{3x3}, i \in \{1,2,3,4\}$ be the square root metrics at the vertices $v_i \in \mathbb{R}^3$ of a tet, such that
$A^2_i = g(v_i)$. We represent a tet given by its four vertices by a 3x4 matrix, i.e.
$$\begin{pmatrix}
  \vertbar & \vertbar & \vertbar & \vertbar \\
  v_1 & v_2 & v_3 & v_4 \\
  \vertbar & \vertbar & \vertbar & \vertbar
\end{pmatrix} \in \mathbb{R}^{3\times 4}.$$
Any point $p$ within the tet can then be represented as
$$p = \alpha v_1 + \beta v_2 + \gamma v_3 + \delta v_4$$
with $\alpha, \beta, \gamma, \delta \ge 0$ and $\alpha + \beta + \gamma + \delta = 1$.
This is a linear transformation between two coordinate systems, which we can write in matrix form as
$$\underbrace{\begin{pmatrix}
  \vertbar & \vertbar & \vertbar & \vertbar \\
  v_1 & v_2 & v_3 & v_4 \\
  \vertbar & \vertbar & \vertbar & \vertbar \\
  1 & 1 & 1 & 1
\end{pmatrix}}_{T^{-1}}
\underbrace{
\begin{pmatrix}
  \alpha \\ \beta \\ \gamma \\ \delta
\end{pmatrix}}_{\lambda} = \underbrace{\begin{pmatrix}
  x \\ y \\ z \\ 1 
\end{pmatrix}}_{p}
\iff
T^{-1}\bm{\lambda} = \bm{p} \iff  \bm{\lambda} = T\bm{p}.
$$
$T$ always exists because $v_1, \dots , v_4$ are linearly independent,
else it would not be a valid tetrahedron.
By denoting $T= \{ t_{ij} \}_{i,j \in \{ 1,\dots, 4\} }$, we can write our barycentric functions as
\begin{align*}
\alpha(x,y,z) &= t_{11}x + t_{12}y + t_{13}z + t_{14} \\
\beta(x,y,z) &= t_{21}x + t_{22}y + t_{23}z + t_{24} \\
\gamma(x,y,z) &= t_{31}x + t_{32}y + t_{33}z + t_{34} \\
\delta(x,y,z) &= t_{41}x + t_{42}y + t_{43}z + t_{44}
\end{align*}
The convex combination
$$A(x,y,z) = \alpha A_1 + \beta A_2 + \gamma A_3 + \delta A_4$$
is the metric prescribed in the tet. To find $\nabla \times A$, let
$A = (A^1, A^2, A^3)$. We will need the derivatives for the curl, so let
$$(A^i_j)_x \triangleq \frac{\partial A^i_j}{\partial x}$$ be the derivative with respect to $x$ of entry $i,j$.
For example, $(A^i_j)_x$ is given by
$$(A^i_j)_x = \alpha_x (A_1)^i_j + \beta_x (A_2)^i_j + \gamma_x (A_3)^i_j + \delta_x (A_4)^i_j = t_{11}(A_1)^i_j + t_{21}(A_2)^i_j + t_{31}(A_3)^i_j + t_{41}(A_4)^i_j$$
If we write $T= (T^1, T^2, T^3, T^4)$ and collect $(A_k)^i_j$ into a vector
$$\bar{A^i_j}=\begin{pmatrix}
  (A_1)^i_j \\ (A_2)^i_j \\ (A_3)^i_j \\ (A_4)^i_j
\end{pmatrix}$$
this can be shortened to $\bar{A^i_j}^{\top}T^1 = (A^i_j)_x$.
Analogously, we get
$$(A^i_j)_y = \bar{A^i_j}^{\top}T^2  \text{ and } (A^i_j)_z = \bar{A^i_j}^{\top}T^3$$
The curl is then given by
$$\nabla \times A^i = \begin{pmatrix}
  (A^i_3)_y - (A^i_2)_z \\
  (A^i_1)_z - (A^i_3)_x \\
  (A^i_2)_x - (A^i_1)_y
\end{pmatrix} =
\begin{pmatrix}
  \bar{A^i_3}^{\top}T^2 - \bar{A^i_2}^{\top}T^3\\
  \bar{A^i_1}^{\top}T^3 - \bar{A^i_3}^{\top}T^1\\
  \bar{A^i_2}^{\top}T^1 - \bar{A^i_1}^{\top}T^2\\
\end{pmatrix}$$
and $\nabla \times A = \nabla \times (A^1, A^2, A^3)$.
Notice that the curl is constant within a tetrahedron.

\section{Recursive subdivision}
Whenever the rotation coefficient $R_{qp}$ is needed between some points $p$ and $q$,
it is unclear ahead of time how many sampling points on the line $\ell$
are needed such that $R_{qp}$ accurately describes how the frame rotates along $\ell$.
We apply a recursive subdivision scheme to recursively sample more points on $\ell$ only where is needed
until sampling more points leads to no noticable improvement anymore (see Figure \ref{fig:recursiveSubdivision}).
We begin by calculating $R_{ab}$ with just the endpoints. This coefficient is then compared to the result if the midpoint was sampled aswell,
so i.e. if
$$\frac{||R_{ab} - R_{am}\cdot R_{mb}||^2_2}{\ell^2}< \varepsilon$$
then no measurable improvement happened. We divide by the length $\ell = ||a-b||_2$ of the segment $\vec{ab}$, because the segments get smaller
and we want the tolerance $\varepsilon$ to remain the same.
This approach has the advantage of only sampling points where there is improvement.
\begin{figure}[htb]
  \centering
  \def\svgwidth{20em}
  \input{figures/recursiveSubdivision.pdf_tex}
  \caption{Points are recursively sampled as midpoints between the line $\vec{ab}$.
  If sampling more points within some line segment leads to noticable improvement, more points are sampled.}
  \label{fig:recursiveSubdivision}
\end{figure}
The algorithm for this simple recursive scheme is shown in below.
\begin{algo*}
  \vbox{
    \small
    \begin{numbertabbing}
      \reset
      xxxx\=xxxx\=xxxx\=xxxx\=xxxx\=xxxx\=MMMMMMMMMMMMMMMMMMM\=\kill
      \textbf{recursiveDivide} $(a,b)$ \label{}\\
      \> $R_1 \gets \op{Rotation(a,b)}$ \label{}\\
      \> $\ell \gets \op{length}(a-b)$ \label{}\\
      \> midpoint $\gets \frac{a+b}{2}$\label{}\\
      \> \textbf{if} $\frac{||R_1 - \mathrm{Rotation}(a,\mathrm{midpoint})\cdot \mathrm{Rotation}(\mathrm{midpoint},b)||^2_2}{\ell^2} < \varepsilon$ \label{}\\
      \> \> return $R_1$ \label{}\\
      \> \textbf{else} \label{}\\
      \> \> return $\op{recursiveDivide}(a,\mathrm{midpoint}) \cdot \op{recursiveDivide}(\mathrm{midpoint}, b)$ \label{}\\
    \end{numbertabbing}
  }
  \caption{Recursive Subdivision}
  \label{alg:recursiveSubdivision}
\end{algo*}


\end{document}

