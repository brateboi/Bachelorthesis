\documentclass[../thesis.tex]{subfiles}
\begin{document}


\chapter{Algorithm for $R$ between two arbitrary points in a mesh}
\cite{Fang23}
Here is an example for how to specify an algorithm in pseudo-code.

% These macros are only used for the pseudo-code example.
\newcommand{\str}[1]{\textsc{#1}}
\newcommand{\var}[1]{\textit{#1}}
\newcommand{\op}[1]{\textsl{#1}}
\def \ifempty#1{\def\temp{#1} \ifx\temp\empty }
\newcommand{\msg}[2]{\ensuremath{\ifempty{#2} [\str{#1}] \else [\str{#1}, {#2}] \fi}}
\newcommand{\tup}[1]{\ensuremath{\langle #1 \rangle}}
\newcommand{\nil}{\ensuremath{\bot}}
\newcommand{\false}{\textsc{false}\xspace}
\newcommand{\true}{\textsc{true}\xspace}

\newcommand\lastts{\var{lastts}\xspace}
\newcommand\nextts{\var{nextts}\xspace}
\newcommand\trusted{\var{trusted}\xspace}
\newcommand\newepoch{\var{newepoch}\xspace}
\newcommand\leader{\var{leader}\xspace}
\newcommand\ts{\var{ts}\xspace}
\newcommand{\CK}{\ensuremath{\mathcal{K}}\xspace}
\newcommand{\CP}{\ensuremath{\mathcal{P}}\xspace}
\newcommand{\CQ}{\ensuremath{\mathcal{Q}}\xspace}


\begin{algo*}
  \vbox{
    \small
    \begin{numbertabbing}
      xxxx\=xxxx\=xxxx\=xxxx\=xxxx\=xxxx\=MMMMMMMMMMMMMMMMMMM\=\kill
      \textbf{State} \label{}\\
      \> \(\lastts \gets 0\): most recently started epoch \label{}\\
      \> \(\nextts \gets 0\): timestamp of the next epoch \label{}\\
      \> \(\newepoch \gets \msg{\nil}{}^n\): list of \str{newepoch} messages\label{}\\
      \\
      \textbf{upon event} \(\op{complain}(p_{\ell})\) \textbf{such that} \( p_{\ell} = \leader(\lastts)\)  \textbf{do} \label{}\\  
      \> \textbf{if} \(\nextts = \lastts\) \textbf{then}\label{}\\
      \> \> \(\nextts \gets \lastts + 1\) \label{}\\
      \> \> send message \(\msg{\str{newepoch}}{\nextts}\) to all \(p_j \in \mathcal{P}\) \label{}\\
      \\
      \textbf{upon} receiving a message
      \(\msg{\str{newepoch}}{\ts}\)  from $p_j$
      \textbf{such that}  \(\ts = \lastts + 1 \) \textbf{do} \label{}\\
      \> \(\newepoch[j] \gets \str{newepoch}\) \label{}\\
      \\
      \textbf{upon exists} \(\ts\) \textbf{such that} \(\{ p_j \in \CP |~\newepoch[j] = \ts \} \in \CK_i\) \textbf{and} \(\nextts = \lastts\) \textbf{do} \label{}\\
      \> \(\nextts \gets \lastts + 1\) \label{}\\
      \> send message \(\msg{\str{newepoch}}{\nextts}\) to all \(p_j \in \CP\) \label{}\\
      \\
      \textbf{upon exists} \(\ts\) \textbf{such that} \(\{ p_j \in \CP |~\newepoch[j] = \ts \} \in \CQ_i\) \textbf{and} \(\nextts > \lastts\) \textbf{do} \label{}\\
      \> \(\lastts \gets \nextts\) \label{}\\
      \> \(\newepoch \gets [\nil]^n\)\label{}\\
      \> \textbf{output} \(\op{startepoch}(\lastts, \leader(\lastts))\) \label{}
    \end{numbertabbing}
  }
  \caption{Byzantine Leader-Based Epoch-Change (process $p_i$).}
  \label{alg:epoch-change}
\end{algo*}



\end{document}