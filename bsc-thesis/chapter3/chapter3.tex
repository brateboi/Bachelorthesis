\documentclass[../thesis.tex]{subfiles}

\begin{document}
\chapter{Connection One-Form $\omega$}
\label{ch:connection}

Recall that our goal is to find a frame field $F:\mathcal{M}\to \mathbb{R}^{3\times3}$,
such that the target parametrization of $\phi$ can sufficiently reduce
the difference between $\nabla \phi$ and $F^{-1}$.
For this, $F^{-1}$ needs to stay as close to being locally integrable
as possible. 
In this chapter, by starting from the local integrability condition,
we formulate a connection one-form $\omega$ which is used to
measure the Dirichlet energy in the new metric $g$, such that
the frames stay close to being $g$-orthonormal ($F^{-1}gF= \Id$).
We follow the steps in \emph{Metric-Driven 3D Frame Field Generation}\cite{Fang23}.

\section{Local integrability}
A vector field $U$ is integrable if and only if $\nabla \times U = 0$,
which means the vector field has vanishing curl \cite{Pap20}.
Although in general it is more complicated \cite{Nieser}, we can think of a frame field $F$ as the composition of 3 vector fields
$$F = \begin{bmatrix}
  \vertbar & \vertbar & \vertbar \\
  F_1 & F_2 & F_3 \\
  \vertbar & \vertbar & \vertbar
\end{bmatrix}$$
where $F_i : \mathbb{R}^3 \to \mathbb{R}^3$ are vector fields.
To achieve local integrability for $F^{-1}$, we therefore want
$$\nabla \times F^{-1} \overset{!}{=}0$$
where the curl is applied to each column.
We can express this more naturally with the language of differential forms:
The curl can be written as the exterior derivative $d$ of a one-form $\alpha$.
A one-form (more generally, a differential form) is closed, if $d\alpha = 0$.
We construct a vector-valued one-form out of our frame field,
given $\bm{p}=(x,y,z)^{\top}$ in Euclidean coordinates
$$\alpha \triangleq F^{-1}d\bm{p} = R^{\top}g^{1/2}d\bm{p}$$
where $d\bm{p} = (dx,dy,dz)^{\top}$ is the common orthonormal one-form basis.
Local integrability of $F^{-1}$ is then formulated as the closedness of $\alpha$, i.e.
$$F^{-1} \text{ locally integrable} \iff \bm{0} = d\alpha$$
Some reformulations yield:
\begin{align*}
  \bm{0} &= d\alpha = d(F^{-1}d\bm{p}) = d(R^{\top}g^{1/2}d\bm{p})\\
  &= dR^{\top} \wedge  (g^{1/2}d\bm{p}) + R^{\top}d(g^{1/2}d\bm{p}) \\
  &= R^{\top}(\omega \wedge (g^{1/2}d\bm{p})+ d(g^{1/2}d\bm{p}))
\end{align*}
which we can simplify to
\begin{equation}\label{eq:oneform}
  \bm{0} = \omega \wedge (g^{1/2}dp)+ d(g^{1/2}dp)
\end{equation}
where the Leibnitz Rule for the exterior derivative is applied
and $\wedge$ is the exterior product between a matrix-valued one-form and a vector-valued one-form,
i.e. one-forms in each component and the matrix vector product uses $\wedge$
as the multiplication, see section \ref{sec:evaluation} for the evaluation.
We further define
$$\omega = RdR^{\top} \in \mathfrak{so}(3)$$
which is an antisymmetric matrix-valued one-form. To see this, 
we differentiate the orthogonality condition of the rotation matrix $R$ (we assume $R$ rotates about the axis $a$ with angle $\theta$):
\begin{align*}
  \Id &= RR^{\top} \\
  d(\Id) &= d(RR^{\top}) \\
  \bm{0} &= dRR^{\top} + RdR^{\top} \\
  \bm{0} &= (RdR^{\top})^{\top} + RdR^{\top} \\
  -(RdR^{\top})^{\top} &= RdR^{\top}
\end{align*}
The Lie algebra $\mathfrak{so}(3)$ consists of all antisymmetric 3x3 matrices and has manifold structure.
Elements of $\mathfrak{so}(3)$ are infinitesimal rotations, that is,
they are tangent to the manifold $\mathrm{SO}(3)$ at the element $\Id$.
Thus, we can use $\omega$ as a connection one-form to do the alignment of frames in
the new metric $g$, i.e. the parallel transport is done with $\omega$ and then compared.

To solve for local integrability, we find $\omega$ such that the one-form $\alpha$ is closed,
then try to match the $\omega$ with $R$. This can be expressed as
\begin{equation}\label{eq:objective}
  \min_{R\in \mathrm{SO}(3)} ||RdR^{\top} - \omega ||^2,
\end{equation}
where $\omega$ is determined by $g$ through a system of linear equations (see \ref{sec:evaluation}).
In general, Eq. \ref{eq:objective} cannot be minimised to zero, therefore we solve we solve
for the nearly integrable 3D rotation field $R$.

\section{Smoothness measure}
Many frame field generation methods rely on maximising smoothness through
the minimization of the Dirichlet energy.
Here, we show that integrability through $\omega$ is closely related to the usual
Dirichlet energy.
We can take Eq. \ref{eq:objective} as a smoothness measure and reformulate, i.e.
\begin{equation}\label{eq:smoothness}
  ||RdR^{\top}-\omega||^2 = ||-dRR^{\top} - \omega||^2 = \cdot||-dRR^{\top}R-\omega R||^2 = ||dR+ \omega R||^2.
\end{equation}
By defining
\begin{equation}
  \mathcal{D}R \triangleq dR + \omega R,
\end{equation}
we can measure the smoothness and the integrability with a single energy $\int_{\mathcal{R}}||\mathcal{D}R||^2$.
Now, when $g$ is constant in euclidean coordinates $(x,y,z)$, Eq. \ref{eq:oneform} tells us
that $\omega = 0$, which reduces the smoothness measure $||\mathcal{D}R||$ to 
$||dR||^2$, which corresponds to the usual Dirichlet energy.
In fact, $\mathcal{D}R$ corresponds to the covariant derivative of $R$ under
the connection $\omega$, which shows that local integrability is related
to the covariant-based Dirichlet energy.
Special care must be taken as we abuse the notation of the covariant derivative. The
covariant derivative $\mathcal{D}$ acts on each column of $R$ separately.
We define 
$$\mathcal{D}R_i \triangleq \nabla_{\dot{\gamma}(t)}R_i \triangleq dR_i + \omega R_i$$
where $dR_i$ is the derivative of each entry with respect to the angle $\theta$ and $\gamma$ is some curve on the manifold.
% The covariant derivative $\mathcal{D}R$ is given by 

\section{Connection evaluation}\label{sec:evaluation}
To find $\omega$, we use equation \ref{eq:oneform}
$$\bm{0} = \omega \wedge (g^{1/2}d\bm{p})+ d(g^{1/2}d\bm{p})$$
and reformulate into a linear system.
We represent the antisymmetric matrix-valued one-form $\omega$
$$\omega = \begin{bmatrix}
 0 & \omega_{12} & -\omega_{31} \\
 -\omega_{12} & 0 & \omega_{23} \\
 \omega_{31} & -\omega_{23} & 0
\end{bmatrix}$$
by $
\begin{bmatrix}
  \omega_{23} & \omega_{31} & \omega_{12}
\end{bmatrix} = 
\begin{bmatrix}
  dx & dy & dz
\end{bmatrix} W$. We write $W=\begin{bmatrix}
  W_1,W_2,W_3
\end{bmatrix}$, $W_i \in \mathbb{R}^3$. That is, $W$ is the matrix with the coefficients for the one-forms, e.g.
$W_1 = \begin{bmatrix}
  (\omega^{23})_1, (\omega^{23})_2, (\omega^{23})_3
\end{bmatrix}^{\top}$
.
Recall, a one-form can be expressed as
$$
\omega_{ij}= (\omega^{ij})_1dx + (\omega^{ij})_2dy + (\omega^{ij})_3dz
$$
So e.g. for $\omega_{23}$ we get $$\omega_{23} = \begin{bmatrix}
  dx & dy & dz
\end{bmatrix}W_1 = (\omega^{23})_1dx + (\omega^{23})_2dy + (\omega^{23})_3dz.$$
We also write $A = g^{1/2}= \begin{bmatrix}
  A^1, A^2, A^3
\end{bmatrix}$.
Starting with the first part of eq. \ref{eq:oneform}, we get
\begin{align*}
  \omega \wedge g^{1/2}dp &= 
  \begin{bmatrix}
    0 & \omega_{12} & -\omega_{31} \\
    -\omega_{12} & 0 & \omega_{23} \\
    \omega_{31} & -\omega_{23} & 0
   \end{bmatrix}
   \bigwedge \begin{bmatrix}
    A^1_1 & A^2_1 & A^3_1 \\ A^1_2 & A^2_2 & A^3_2 \\ A^1_3 & A^2_3 & A^3_3 
  \end{bmatrix}
  \begin{bmatrix}
    dx \\ dy \\ dz
  \end{bmatrix} \\
  &= \begin{bmatrix}
    +\omega_{12} \wedge (A^1_2dx + A^2_2dy + A^3_2dz) - \omega_{31} \wedge (A^1_3dx + A^2_3dy + A^3_3dz) \\
    -\omega_{12} \wedge (A^1_1dx + A^2_1dy + A^3_1dz) + \omega_{23} \wedge (A^1_3dx + A^2_3dy + A^3_3dz) \\
    +\omega_{31} \wedge (A^1_1dx + A^2_1dy + A^3_1dz) - \omega_{23} \wedge (A^1_2dx+ A^2_2dy + A^3_2dz)
  \end{bmatrix}
\end{align*}
It will get really messy if we calculate each component here, so let us calculate one component separately first:
\begin{align*}
  \omega_{ij} \wedge (A^1_kdx + A^2_kdy + A^3_kdz) &=
  ((\omega^{ij})_1dx + (\omega^{ij})_2dy + (\omega^{ij})_3dz) \wedge (A^1_kdx + A^2_kdy+A^3_kdz) \\
  &= (\omega^{ij})_1 A^2_k dx\wedge dy + (\omega^{ij})_1 A^3_k dx \wedge dz \\
  &+ (\omega^{ij})_2 A^1_k dy\wedge dx + (\omega^{ij})_2 A^3_k dy \wedge dz \\
  &+ (\omega^{ij})_3 A^1_k dz\wedge dx + (\omega^{ij})_3 A^2_k dz \wedge dy \\
  &= ((\omega^{ij})_1A^2_k - (\omega^{ij})_2A^1_k)dx \wedge dy \\
  &+ ((\omega^{ij})_2A^3_k - (\omega^{ij})_3A^2_k)dy \wedge dz \\
  &+ ((\omega^{ij})_3A^1_k - (\omega^{ij})_1A^3_k)dz \wedge dx
\end{align*}
where we use the fact that $dx \wedge dx = 0$ and $dx \wedge dy = -dy \wedge dx$.
We can clean up the above expression using the cross product:
\begin{align*}
  \begin{bmatrix}
    ((\omega^{ij})_2A^3_k - (\omega^{ij})_3A^2_k) \\
    ((\omega^{ij})_3A^1_k - (\omega^{ij})_1A^3_k) \\
    ((\omega^{ij})_1A^2_k - (\omega^{ij})_2A^1_k) 
  \end{bmatrix}^{\top}
  \begin{bmatrix}
    dy \wedge dz \\
    dz \wedge dx \\
    dx \wedge dy 
  \end{bmatrix} = 
  \begin{bmatrix}
    W_i \times A^k
  \end{bmatrix}^{\top}
  \begin{bmatrix}
    dy \wedge dz \\
    dz \wedge dx \\
    dx \wedge dy 
  \end{bmatrix}
\end{align*}
We use the fact that $\begin{bmatrix}
  A_1, A_2, A_3 
\end{bmatrix} = \begin{bmatrix}
  A^1, A^2, A^3
\end{bmatrix}$ because $A$ is symmetric, and $W_1$ corresponds to $\omega_{23}$, $W_2$ to $\omega_{31}$ and $W_3$ to $\omega_{12}$.
We continue with the second part of eq. \ref{eq:oneform}:
\begin{align*}
  d(g^{1/2}d\bm{p})= d(Ad\bm{p}) = d\left(\begin{bmatrix}
    A^1_1 & A^2_1 & A^3_1 \\
    A^1_2 & A^2_2 & A^3_2 \\
    A^1_3 & A^2_3 & A^3_3
  \end{bmatrix}
  \begin{bmatrix}
    dx \\ dy \\ dz
  \end{bmatrix}
  \right)
\end{align*}
Again, we can do this separately for each row (we use the fact that the exterior derivative $d$ is the ordinary differential for a smooth function):
\begin{align*}
  & \hspace{1.35em} d(A^1_k dx + A^2_k dy + A^3_k dz) \\
  &= dA^1_k \wedge dx + dA^2_k \wedge dy + dA^3_k \wedge dz \\
  &= \frac{\partial A^1_k}{\partial x}dx \wedge dx + \frac{\partial A^1_k}{\partial y}dy \wedge dx + \frac{\partial A^1_k}{\partial z}dz \wedge dx \\
  &+ \frac{\partial A^2_k}{\partial x}dx \wedge dy + \frac{\partial A^2_k}{\partial y}dy \wedge dy + \frac{\partial A^2_k}{\partial z}dz \wedge dy \\
  &+ \frac{\partial A^3_k}{\partial x}dx \wedge dz + \frac{\partial A^3_k}{\partial y}dy \wedge dz + \frac{\partial A^3_k}{\partial z}dz \wedge dz \\
  &= \left( \frac{\partial A^3_k}{\partial y} - \frac{\partial A^2_k}{\partial z} \right) dy \wedge dz + \left( \frac{\partial A^1_k}{\partial z} - \frac{\partial A^3_k}{\partial x} \right) dz \wedge dx + \left( \frac{\partial A^2_k}{\partial x} - \frac{\partial A^1_k}{\partial y} \right) dx \wedge dy \\
  &= (\nabla \times A_k)^{\top} \begin{bmatrix}
    dy \wedge dz \\ dz \wedge dx \\ dx \wedge dy
  \end{bmatrix} 
\end{align*}
Finally, we can put everything together:
\begin{align*}
  & \hspace{1em}\bm{0} = \begin{bmatrix}
    (W_3 \times A^2 - W_2 \times A^3 + \nabla \times A^1)^{\top} \\
    (W_1 \times A^3 - W_3 \times A^1+ \nabla \times A^2)^{\top} \\
    (W_2 \times A^1 - W_1 \times A^2 + \nabla \times A^3)^{\top}
  \end{bmatrix}\begin{bmatrix}
    dy \wedge dz \\ dz \wedge dx \\ dx \wedge dy
  \end{bmatrix} \\
  & \iff 
  \begin{bmatrix}
    (W_2 \times A^3 - W_3 \times A^2)^{\top}  \\ 
    (W_3 \times A^1 - W_1 \times A^3)^{\top}  \\
    (W_1 \times A^2 - W_2 \times A^1)^{\top}
  \end{bmatrix}
  \begin{bmatrix}
    dy \wedge dz \\ dz \wedge dx \\ dx \wedge dy
  \end{bmatrix}
  =
  \begin{bmatrix}
    (\nabla \times A^1)^{\top} \\
    (\nabla \times A^2)^{\top} \\
    (\nabla \times A^3)^{\top} \\
  \end{bmatrix}
  \begin{bmatrix}
    dy \wedge dz \\ dz \wedge dx \\ dx \wedge dy
  \end{bmatrix}
\end{align*}
Take the curl to the other side and switch order on the left-hand side to cancel the $-1$.
As we are only interested in the 9 components of $W$, we omit the
two-form basis and transform into a 9x9 linear system for $W$.
We define $A_{\times}$ and $\vecop(\cdot)$ as
$$A_{\times} = \begin{bmatrix}
  0 & -A^3_{\times} & A^2_{\times} \\
  A^3_{\times} & 0 & -A^1_{\times} \\
  -A^2_{\times} & A^1_{\times} & 0
\end{bmatrix},
\vecop(W)= \begin{bmatrix}
  W_1 \\ W_2 \\ W_3
\end{bmatrix}$$
with $A^i_{\times}$ defined as
$$A^i_{\times} = \begin{bmatrix}
  A^i_1 & A^i_2 & A^i_3
\end{bmatrix}_{\times} =
\begin{bmatrix}
  0 & -A^i_3 & A^i_2 \\
  A^i_3 & 0 & -A^i_1 \\
  -A^i_2 & A^i_1 & 0
\end{bmatrix}$$ and $\vecop(\cdot)$ turns a 3x3-matrix into a 9x1-vector by stacking the columns.
With these two definitions, we can transform the above equality into a linear system
$$A_{\times} \vecop(W) = \vecop (\nabla \times A)$$
where $\nabla \times A$ is just the curl applied to each column. This transformation can be checked by laboriously plugging in the definitions and comparing the coefficients.
With tedious brute-force calculations, one can show that $\det(A_{\times})=-2\det(A)^3=-2\det(g)^{3/2}<0$, which
means this is a linear system that is solvable and can be used to calculate $W$ at a point.




\end{document}